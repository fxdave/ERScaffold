\begin{center}
\section*{\textbf{\Large \MakeUppercase{\textrm{Tartalmi összefoglaló}}}}
\end{center}


\begin{justify}
A szakdolgozat témája egy webalapú ER modellező készítése. Ezen belül is egy megfelelő eszköz biztosítása a rapid-prototyping fejlesztési módszer segítésére, első sorban web alapú alkalmazásokhoz.

A webes alkalmazások nagy része tartalmaz olyan közös elemeket, amelyek lehetővé teszik, hogy akár kevés információval is, de generálhatóak legyenek.

A szakdolgozatban bemutatott program egy módosított Entity Relationship modellt használ a legenerálandó erőforrások definiálására. Ehhez a modellhez biztosít egy tervező felületet, hogy a generátor használata a lehető legkényelmesebb, és leggyorsabb legyen.

Az átláthatóság érdekében a generátor sablonjai csomagokba lettek rendezve, melyeken belül találhatók továbbá a követelményfájlok, amelyek leírják, hogy milyen követelményeknek felelhet meg az aktuális sabloncsomag. Ezen követelmények mindegyike opcionális, így lehetőséget ad arra is hogy a nem kívánt funkciókat ne vigye bele a rendszerbe.

A program  biztosít lehetőséget saját sabloncsomagok létrehozására és a meglévők módosítására is.

A kódgenerátorok nagy hátránya, hogy az absztrakt megfogalmazás már nem módosítható anélkül, hogy a kódot újra kelljen generálni. A szakdolgozat keretében elkészített alkalmazás ezen nem változtat, viszont ad egy megoldást erre a problémára a git verziókezelő rendszer használatával. 

A program segítségével lehetőség nyílik arra, hogy ne csak egyszerű, hanem komplex szerver- és ezzel együtt kliens oldali kódokat tudjunk generálni, ezzel utat nyitva a jól strukturált, könnyen módosítható, webes alkalmazások felé.
\end{justify}

\vspace{2cm}

{\bf Kulcsszavak:} {\it scaffolding, web, kódgenerátor, rapid-prototyping, ER, modellező}
